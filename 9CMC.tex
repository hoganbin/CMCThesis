\documentclass[11pt,twoside]{ctexart}
\usepackage{mtpro2}
\usepackage{CJKnumb}
\usepackage{amsmath,amssymb}
\usepackage{calc}
\usepackage{intcalc}
\usepackage{ifthen}
\usepackage{zref-user}
\usepackage{zref-lastpage}
\usepackage{makecell}
\usepackage{interfaces-makecell}
\usepackage{dashrule}
\usepackage{parskip}
\usepackage[paperwidth=195mm,paperheight=270mm,left=30mm,right=25mm,top=20mm,bottom=20mm,includefoot]{geometry}
\usepackage{enumerate}
\usepackage{fancyhdr}
\usepackage{tcolorbox}
\usepackage{lastpage}
\pagestyle{fancy}

%用到的长度变量
\newlength{\wot}
\newlength{\wol}
\newlength{\gmw}
\newlength{\dl}

%长度变量的初始值
\settowidth{\wot}{复核人}
\setlength{\wol}{0.3pt}
\setlength{\gmw}{6em}
\setlength{\dl}{10em}

%页眉设置开始
\renewcommand{\headrulewidth}{0pt}

%装订线开始
\fancyheadoffset[OL,ER]{\gmw}
\fancyhead[OL]{
\ifnum\intcalcMod{\value{page}}{4}=1
\rotatebox{90}
{\begin{minipage}{1.1\textheight}
\begin{center}
省市:\rule[-.2ex]{\dl}{\wol} 学校:\rule[-.2ex]{\dl}{\wol}  姓名:\rule[-.2ex]{\dl}{\wol} 准考证号:\rule[-.2ex]{\dl}{\wol}\\
\tiny \hdashrule[-3ex]{\textheight}{\wol}{3pt}\\[\smallskipamount]
\makebox[0.6\textheight][s]{装订线内不要答题}\\[-3\smallskipamount]
\hdashrule[-3ex]{\textheight}{\wol}{3pt}
\end{center}
\end{minipage} }
\fi
}
\fancyhead[ER]{
\ifnum\intcalcMod{\value{page}}{4}=0
\rotatebox{-90}
{\begin{minipage}{1.1\textheight}
\begin{center}
\tiny \hdashrule[-3ex]{\textheight}{\wol}{3pt}\\[\smallskipamount]
\makebox[0.6\textheight][s]{装订线内不要答题}\\[-3\smallskipamount]
\hdashrule[-3ex]{\textheight}{\wol}{3pt}
\end{center}
\end{minipage} }
\fi
}
%装订线结束
%页眉设置结束
%页脚设置开始
\renewcommand{\footrulewidth}{\wol}
\fancyfoot[C]{\heiti 微信公众号之数学的情怀\quad 第{\bf\thepage} 页 (共~{\bf\pageref{LastPage}}~页)}
%\fancyfoot[C]{\large{{ \textbf{微信公众号之数学的情怀}}\qquad 共\zpageref{LastPage}页\quad 第\thepage 页}}
\newcounter{ns}
\newcounter{ts}
\newcounter{nq}
\newcommand{\wns}{\stepcounter{ns}\CJKnumber{\thens}、}
\newcommand{\wq}{\stepcounter{nq}\thenq.}

%大题前计分表格
\newcommand{\tbs}{\begin{tabular}{|c|c|c|}\hline \makebox[\wot]{得分}&\makebox[\wot]{评卷人}&\makebox[\wot]{复核人}\\ \hline
 & &\\ \hline\end{tabular}}

%排版大题前计分表格,序号,题型,大题说明
\newcommand{\ws}[2]{\raisebox{-1ex}{\begin{minipage}[b]{4.6\wot}\tbs\end{minipage}}
\begin{minipage}[t]{\textwidth-6\wot} {\heiti \wns #1 } #2 \end{minipage} }
\makeatletter
\zref@newprop{totalsections}[3]{\arabic{ns}}
\zref@addprop{LastPage}{totalsections}
\AtBeginDocument{
\setcounter{ts}{\zref@extractdefault{LastPage}{totalsections}{3}} }
\makeatother
\linespread{1.618}
\newcommand{\D}{\,\mathrm{d}}
\newcommand{\E}{\mathrm{e}}
\newcommand{\dlim}{\displaystyle \lim }
\newcommand{\dint}{\displaystyle \int }
\newcommand{\sets}[1]{\{ #1 \}}

\setCJKfamilyfont{huawenxingkai}{华文行楷} \newcommand*{\xingkai}{\CJKfamily{huawenxingkai}}%华文行楷
\newcommand{\dis}{\displaystyle}
\newcommand{\Rank}{\mathrm{Rank}\mspace{1mu}}
\newcommand{\Sign}{\mathrm{Sign}\mspace{1mu}}
\newcommand{\rd}{\mspace{1mu}\mathrm{d}}
\newcommand{\diag}{\mathrm{diag}\mspace{1mu}}
\newcommand{\tr}{\mathrm{tr}\mspace{1mu}}
\newcommand{\var}{\mathrm{var}}
\renewcommand{\Re}{\operatorname{Re}}
\renewcommand{\Im}{\operatorname{Im}}

%直立积分号,需要mathabx宏包
\makeatletter
\def\upintkern@{\mkern-7mu\mathchoice{\mkern-2mu}{}{}{}}
\def\upintdots@{\mathchoice{\mkern-4mu\@cdots\mkern-4mu}%
	{{\cdotp}\mkern1.5mu{\cdotp}\mkern1.5mu{\cdotp}}%
	{{\cdotp}\mkern1mu{\cdotp}\mkern1mu{\cdotp}}%
	{{\cdotp}\mkern1mu{\cdotp}\mkern1mu{\cdotp}}}
\newcommand{\upiint}{\DOTSI\protect\UpMultiIntegral{2}}
\newcommand{\upiiint}{\DOTSI\protect\UpMultiIntegral{3}}
\newcommand{\upiiiint}{\DOTSI\protect\UpMultiIntegral{4}}
\newcommand{\upidotsint}{\DOTSI\protect\UpMultiIntegral{0}}
\newcommand{\UpMultiIntegral}[1]{%
	\edef\ints@c{\noexpand\upintop
		\ifnum#1=\z@\noexpand\upintdots@\else\noexpand\upintkern@\fi
		\ifnum#1>\tw@\noexpand\upintop\noexpand\upintkern@\fi
		\ifnum#1>\thr@@\noexpand\upintop\noexpand\upintkern@\fi
		\noexpand\upintop
		\noexpand\ilimits@
	}%
	\futurelet\@let@token\ints@a
}
\makeatother

\DeclareFontFamily{U}{mathx}{\hyphenchar\font45}
\DeclareFontShape{U}{mathx}{m}{n}{
	<->s * [0.8]
	mathx10
}{}
\DeclareSymbolFont{mathx}{U}{mathx}{m}{n}
\DeclareFontSubstitution{U}{mathx}{m}{n}
\DeclareMathSymbol{\upintop}{\mathop}{mathx}{'273}
%\DeclareMathSymbol{\upiint}{\mathop}{mathx}{'274}
%\DeclareMathSymbol{\upiiint}{\mathop}{mathx}{'275}
\DeclareMathSymbol{\upointop}{\mathop}{mathx}{'276}
\DeclareMathSymbol{\upoiint}{\mathop}{mathx}{'277}
\makeatletter
\newcommand{\upint}{\DOTSI\upintop\ilimits@}
\newcommand{\upoint}{\DOTSI\upointop\ilimits@}

%分数线延长mdugm
\newcommand{\zfrac}[2]{\dfrac{{\raisebox{-0.7mm}{$#1$}}}{\;{\raisebox{0.2mm}{$#2$}}\;}}
\newcommand{\bfrac}[2]{\dfrac{{\raisebox{-0.7mm}{$#1$}}}{{\raisebox{0.2mm}{$#2$}}}}

%======================
%试卷头开始
\begin{document}
%试卷标题开始
\begin{center}\vspace{3mm}
      {\xingkai \Large 第九届全国大学生数学竞赛预赛参考答案}\\[0.8mm]
      { $\left(\text{非数学类, 2017年10月28日}\right)$}\\
\end{center}

%试卷标题结束

%输出"绝密"字样
{\vspace{-1.3mm}\heiti 绝密$\bigstar$启用前}\\[-4\bigskipamount]\\[-12mm]
\begin{center}
\vspace*{2mm}
(16数学$-$胡八一)\\[3mm]
 {考试形式:\underline{~闭卷~}~\hspace{2mm}考试时间:\underline{~~150~~}分钟~\hspace{2mm}满分:~\underline{~~100~~}~分}\\

%根据大题数目自动生成计分总表
\newcounter{tc}
\newcounter{tcsr}
\setcounter{tc}{\value{ts}+3}
\setcounter{tcsr}{\value{tc}-1}
\arrayrulewidth=2\wol 

\vspace*{3.5mm}
\begin{tabular}{|m{3em}<{\centering}|*{7}{m{3.5em}<{\centering}|}}\hline
         题~号 & 一 & 二 & 三  & 四 & 五   &总~~分 \\\hline
		满~分 & 42 & 14 & 14  & 15 & 15   &\raisebox{0.4em}{100}\rule{0pt}{8mm}\\\hline
		得~分 &    &    &      &    &    &    \rule{0pt}{8mm} \\\hline
	\end{tabular}
	\\\vspace*{-1.5mm}
	\begin{equation*}
	\begin{aligned}
	\mbox{注意:}
	&1.\,\mbox{所有答题都须写在试卷密封线右边,写在其他纸上一律无效}.\hspace{12.0cm}\\
	&2.\,\mbox{密封线左边请勿答题,密封线外不得有姓名及相关标记}.\\
	&3.\,\mbox{如答题空白不够,可写在当页背面,并标明题号}.\\[-2mm]
	\end{aligned}
	\end{equation*}	
\end{center}

%试卷头结束

\addvspace{1\bigskipamount}

\ws{ \textbf{填空题}}{( \textbf{本题满分42分})\\}\\\\
\wq 已知可导函数$f(x)$满足$f(x)\cos x+2\displaystyle\upint_0^x f(t)\sin t\text{d}x=x+1$,则$f(x)=$\underline{\hspace{3em}}.

【答案】$\sin x+\cos x$.

【解析】两边同时对$x$求导
\[f'(x)\cos x+f(x)\sin x=1\Longrightarrow f'(x)+f(x)\tan x=\sec x.\]
由常数变易法,从而
\begin{align*}
f(x)&=\mathrm{e}^{-\upint\tan x\mathrm{d}x}\Big(\upint\sec x\mathrm{e}^{\upint\tan x\mathrm{d}x}\mathrm{d}x+C\Big)\\
&=\mathrm{e}^{\ln\cos x}\Big(\upint\zfrac{1}{\cos x}\mathrm{e}^{-\ln\cos x}\mathrm{d}x+C\Big)\\
&=\cos x\Big(\upint\zfrac{1}{\cos^2x}\mathrm{d}x+C\Big)\\
&=\cos x(\tan x+C)=\sin x+C\cos x
\end{align*}

由于$f(0)=1$,故$f(x)=\sin x+\cos x.$


\wq 极限$\displaystyle\lim_{n\rightarrow 0}\sin^2(\pi \sqrt{n^2+n})=$\underline{\hspace{3em}}.

【答案】1

【解析】
\begin{align*}
\lim_{n\rightarrow 0}\sin^2(\pi\sqrt{n^2+n})
&=\lim_{n\rightarrow 0}\sin^2(\pi\sqrt{n^2+n}-n\pi)\\
&=\lim_{n\rightarrow 0}\sin^2\Big(\zfrac{n\pi}{\sqrt{n^2+n}+n}\Big)=1
\end{align*}

\wq 设$w=f(u,v)$具有二阶连续偏导数,且$u=x-cy$,$v=x+cy$,其中$c$为非零常数。则$w_{xx}-\zfrac{1}{c^2}w_{yy}=$\underline{\hspace{3em}}.

【答案】$4f_{12}$.

【解析】\[w_x=f_1+f_2,w_{xx}=f_{11}+2f_{12}+f_{22},w_y=c(f_2-f_1),\]
\[
w_{yy}=c\zfrac{\partial}{\partial x}(f_2-f_1)=c(cf_{11}-cf_{12}-cf_{21}+cf_{22})=c^2(f_{11}-2f_{12}+f_{22}).
\]

所以
\[
w_{xx}-\zfrac{1}{c^2}w_{yy}=4f_{12}.
\]

\wq 设$f(x)$有二阶导数连续,且$f(0)=f'(0)=0$,$f''(0)=6$,则$\displaystyle\lim_{n\rightarrow0}\zfrac{f(\sin^2x)}{x^4}=$\underline{\hspace{3em}}.

【答案】3

【解析】$f(x)$在$x=0$处的泰勒展开式
\[f(x)=f(0)+f'(0)x+\zfrac{1}{2}f''(\xi)x^2,\]
所以
\[f(\sin^2x)=\zfrac{1}{2}f''(\xi)\sin^4x,\]
于是
\[\displaystyle\lim_{n\rightarrow0}\zfrac{f(\sin^2x)}{x^4}=\lim_{n\rightarrow0}\zfrac{\zfrac{1}{2}f''(\xi)\sin^4x}{x^4}=3.\]

\wq 不定积分$\displaystyle\upint\zfrac{\mathrm{e}^{-\sin x}\sin 2x}{(1-\sin x)^2}\mathrm{d}x=$\underline{\hspace{3em}}.

【答案】$\zfrac{2\mathrm{e}^{-\sin x}}{1-\sin x}+C.$

【解析】令$\sin x=v$,则
\begin{align*}
I &=2\upint\zfrac{v\mathrm{e}^{-v}}{(1-v)^2}\mathrm{d}v\\
&=2\upint\zfrac{\mathrm{e}^{-v}}{v-1}\mathrm{d}v+2\upint\zfrac{\mathrm{e}^{-v}}{(v-1)^2}\mathrm{d}v\\
&=2\upint\zfrac{\mathrm{e}^{-v}}{v-1}\mathrm{d}v-2\upint\mathrm{e}^{-v}\mathrm{d}\Big(\zfrac{1}{v-1}\Big)\\
&=2\upint\zfrac{\mathrm{e}^{-v}}{v-1}\mathrm{d}v-2\Big(\zfrac{\mathrm{e}^{-v}}{v-1}+\upint\zfrac{\mathrm{e}^{-v}}{v-1}\mathrm{d}v\Big)\\
&=-\zfrac{2\mathrm{e}^{v}}{v-1}+C=\zfrac{2\mathrm{e}^{-\sin x}}{1-\sin x}+C
\end{align*}

\wq 记曲面$z^2=x^2+y^2$和$z=\sqrt{4-x^2-y^2}$围成空间区域$V$,则三重积分$\displaystyle\upiiint_V z\mathrm{d}x\mathrm{d}y\mathrm{d}z\\
=$\underline{\hspace{3em}}.

【答案】$2\pi$.

【解析】使用球面坐标
\begin{align*}
I&=\upiiint_V z\mathrm{d}x\mathrm{d}y\mathrm{d}z\\
&=\upint_0^{2\pi}\mathrm{d}\theta\upint_0^{\pi/4}\mathrm{d}\varphi\upint_0^2\rho\cos \varphi\cdot\rho^2\sin\varphi\mathrm{d}\rho\\
&=2\pi\cdot\zfrac{1}{2}\sin^2\varphi\Big|_0^{\pi/4}\Big.\cdot\zfrac{1}{4}\rho^4\Big|_0^2\Big.\\
&=2\pi
\end{align*}

\ws{ \textbf{解答题}}{(\textbf{本题满分14分})\\}\\\\
设二元函数$f(x,y)$在平面上有连续的二阶导数,对任意角$\alpha$,定义一元函数
\[g_{\alpha}(t)=f(t\cos\alpha,t\sin\alpha).\]
若对任何$\alpha$都有$\zfrac{\mathrm{d}g_{\alpha}(0)}{\mathrm{d}t}=0$且$\zfrac{\mathrm{d}^2g_{\alpha}(0)}{\mathrm{d}t^2}>0$。证明:$f(0,0)$是$f(x,y)$的极小值。

【证明】由于
\[\zfrac{\mathrm{d}g_{\alpha}(0)}{\mathrm{d}t}=(f_x,f_y)_{(0,0)}
\bigg(\begin{array}{@{}l@{}}
\cos\alpha\\
\sin\alpha
\end{array}\bigg)=0
\]
对一切$\alpha$成立,故$(f_x,f_y)_{(0,0)}=(0,0)$,即$(0,0)$是$f(x,y)$的驻点。

\hfill\dotfill 4分

记
\[H_f=(x,y)=\bigg(\begin{array}{@{}rl@{}}
f_{xx}&f_{xy}\\
f_{yx}&f_{yy}
\end{array}\bigg)=0,\]
则
\begin{align*}
\zfrac{\mathrm{d}^2g_{\alpha}(0)}{\mathrm{d}t^2}
&=\zfrac{\mathrm{d}}{\mathrm{d}t}\bigg[(f_x,f_y)\bigg(\begin{array}{@{}l@{}}
\cos\alpha\\
\sin\alpha
\end{array}\bigg)\bigg]_{(0,0)}\\
&=(\cos\alpha,\sin\alpha)H_f(0,0)\bigg(\begin{array}{@{}ll@{}}
f_{xx}&f_{xy}\\
f_{yx}&f_{yy}
\end{array}\bigg)>0
\end{align*}
\hfill\dotfill 10分\\
上式对任何单位向量$(\cos\alpha,\sin\alpha)$成立,故$H_f(0,0)$是一个正定矩阵,而$f(0,0)$是$f(x,y)$的极小值。

\hfill\dotfill 14分

\newpage

\ws { \textbf{解答题}}{( \textbf{本题满分14分})\\}\\\\
设曲线$\Gamma$为曲线
\[x^2+y^2+z^2=1,x+z=1,x\geqslant0,y\geqslant0,z\geqslant0\]
上从点$A(1,0,0)$到点$B(0,0,1)$的一段,求曲线积分$I=\displaystyle\upint_{\Gamma}y\mathrm{d}x+z\mathrm{d}y+x\mathrm{d}z.$

【解析】记$\Gamma_1$为从$B$到$A$的直线段,则$x=t,y=0,z=1-t,0\leqslant t\leqslant 1,$
\[\upint_{\Gamma_1}y\mathrm{d}x+z\mathrm{d}y+x\mathrm{d}z=\upint_0^1t\mathrm{d}(1-t)=-\zfrac{1}{2}.\]
\hfill\dotfill 4分

设$\Gamma$和$\Gamma_1$围成的平面区域$\sum$,方向按右手法则。由Stokes公式得到
\begin{align*}
\Big(\upint_{\Gamma}+\upint_{\Gamma_1}\Big)y\mathrm{d}x+z\mathrm{d}y+x\mathrm{d}z
&=\upiint_{\sum}
\begin{vmatrix}
\mathrm{d}y\mathrm{d}z&\mathrm{d}z\mathrm{d}x&\mathrm{d}x\mathrm{d}y\\
\zfrac{\partial}{\partial x}&\zfrac{\partial}{\partial y}&\zfrac{\partial}{\partial z}\\
y&z&x
\end{vmatrix}
\\
&=-\upiint_{\sum}\mathrm{d}y\mathrm{d}z+\mathrm{d}z\mathrm{d}x+\mathrm{d}x\mathrm{d}y.
\end{align*}
\hfill\dotfill 8分\\
右边三个积分都是$\sum$在各个坐标面上的投影面积,而$\sum$在$xOz$面上的投影面积为零。故
\[I+\upint_{\Gamma_1}=-\upiint_{\sum}\mathrm{d}y\mathrm{d}z+\mathrm{d}x\mathrm{d}y.\]
曲线$\Gamma$在$xOy$面上投影的方程为$\zfrac{\Big(x-\zfrac{1}{2}\Big)^2}{\Big(\zfrac{1}{2}\Big)^2}+\zfrac{y^2}{\Big(\zfrac{1}{\sqrt{2}}\Big)^2}=1.$
\hfill\dotfill 12分\\
又该投影(半个椭圆)的面积为$\upiint_{\sum}\mathrm{d}x\mathrm{d}y=\zfrac{\pi}{4\sqrt{2}},$同理$\upiint_{\sum}\mathrm{d}y\mathrm{d}z=\zfrac{\pi}{4\sqrt{2}}.$

所以
\[I=\zfrac{1}{2}-\zfrac{\pi}{2\sqrt{2}}.\]
\hfill\dotfill 14分

 
 
\newpage
\ws {\textbf{解答题}}{(\textbf{ 本题满分15分})\\}\\\\
设函数$f(x)>0$且在实轴上连续,若对任意实数$t$,有
\[\upint_{-\infty}^{+\infty}\mathrm{e}^{-|t-x|}f(x)\mathrm{d}x\leqslant 1,\]
证明:$\forall a,b,a<b$,有
\[\upint_a^bf(x)\mathrm{d}x\leqslant\zfrac{b-a+2}{2}.\]

【证明】由于$\forall a,b,a<b$,有$\upint_{a}^{+b}\mathrm{e}^{-|t-x|}f(x)\mathrm{d}x\leqslant\upint_{-\infty}^{+\infty}\mathrm{e}^{-|t-x|}f(x)\mathrm{d}x\leqslant 1,$

因此
\[\upint_a^b\mathrm{d}t\upint_a^b\mathrm{e}^{-|t-x|}f(x)\mathrm{d}x\leqslant b-a.\]
\hfill\dotfill 4分\\
然而
\[\upint_a^b\mathrm{d}t\upint_a^b\mathrm{e}^{-|t-x|}f(x)\mathrm{d}x=\upint_a^bf(x)\Big(\upint_a^b\mathrm{e}^{-|t-x|}\mathrm{d}t\Big)\mathrm{d}x,\]
其中
\[\upint_a^b\mathrm{e}^{-|t-x|}\mathrm{d}t=\upint_a^x\mathrm{e}^{t-x}\mathrm{d}t+\upint_x^b\mathrm{e}^{x-t}=2-\mathrm{e}^{a-x}-\mathrm{e}^{x-b}.\]
这样就有
\[
\upint_a^bf(x)(2-\mathrm{e}^{a-x}-\mathrm{e}^{x-b})\mathrm{d}x\leqslant b-a.\eqno{(*)}
\]
\hfill\dotfill 10分\\
即
\[\upint_a^bf(x)\mathrm{d}x\leqslant\zfrac{b-a}{2}+\zfrac{1}{2}\Big[\upint_a^b\mathrm{e}^{a-x}f(x)\mathrm{d}x+\upint_a^b\mathrm{e}^{x-b}f(x)\mathrm{d}x\Big].\]
注意到
\[\upint_a^b\mathrm{e}^{a-x}f(x)\mathrm{d}x=\upint_a^b\mathrm{e}^{-|a-x|}f(x)\mathrm{d}x\leqslant 1\textbf{,}\upint_a^b\mathrm{e}^{x-b}f(x)\mathrm{d}x\leqslant 1.\]
\hfill\dotfill 13分\\
把以上两个式子代入($*$),即得结论。

\hfill\dotfill 15分

\mbox{}

%试卷正文结束
\end{document}
